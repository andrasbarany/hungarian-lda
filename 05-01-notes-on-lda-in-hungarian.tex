% arara: xelatex: { options: [ '-no-pdf' ] }
% arara: biber if found('log', 'run Biber on the file')
% arara: makeglossaries if changed('glo')
% arara: xelatex: { options: [ '-no-pdf' ] } if found('log', 'Please rerun')
% arara: xelatex
\input{\string~/Cloud/LaTeX/Templates/article-preamble.tex}

\addbibresource{legenda.bib}

\usepackage{xurl}

\newleipzig{MNSZ}{MNSZ}{Magyar Nemzeti Szövegtár (Hungarian National Corpus)}

\title{Notes on long-distance agreement in Hungarian}

\begin{document}

%\maketitle

\thispagestyle{plain}
\noindent{\Large\textsf{Notes on long-distance agreement in Hungarian}}\\[.2cm]
\noindent\textsf{András Bárány}

\section{Introduction}\label{sec:introduction}

Infinitival complements in Hungarian appear with different classes of verbs. It
is usually said that verbs that can select for \Acc{} objects
(\enquote{transitive} verbs) can agree with the object an infinitival
complement, but verbs that do not select for \Acc{} objects
(\enquote{intransitive} verbs) cannot
\parencite{EKiss1987,EKiss1989,KalmanCetal1989,KenVagFeny1998,EKiss2002,EKissvanRie2004b,denDikken2004,Coppock2012d,Szecsenyi2017,SzeSze2018}.
%
The construction in question is shown in~\eqref{ex:inf-compl}, with examples
in~\eqref{ex:itv-inf} and~\eqref{ex:tv-inf}.\footnote{Abbreviations:
\printglosses}

\ex\label{ex:inf-compl}\textbf{Matrix verb with infinitival complement}\\
    {}[ \dots{} finite verb [\tss{\Inf} infinitive (object-\Acc{}) ]]
\xe

\pex\label{ex:itv-inf}
    \a
    \begingl
        \glpreamble \textbf{Intransitive matrix verb, intransitive infinitive}//
        \gla 	János igyekez-ett \nogloss{[\tss{\Inf}} bemen-ni \nogloss{]}.//
        \glb 	János strive-\Tsg.\Pst{} enter-\Inf{}//
        \glft 	\enquote*{János strove to enter.}//
    \endgl
    \a
    \begingl
        \glpreamble \textbf{Intransitive matrix verb, transitive infinitive}//
        \gla 	Anna igyekez-ett \nogloss{[\tss{\Inf}} meg-tanul-ni a vers-et \nogloss{]}.//
        \glb 	Anna strive-\Tsg.\Pst{} \Vm-learn-\Inf{} the poem-\Acc{}//
        \glft 	\enquote*{Anna strove to learn the poem.}\trailingcitation{\parencite[33]{KenVagFeny1998}}//
    \endgl
    \a
    \begingl
        \glpreamble \textbf{Intransitive matrix verb, transitive infinitive}//
        \gla 	Igyekez-lek meglátogat-ni (téged).//
        \glb 	{make effort}-\Fsg.\Sbj>\Second.\Obj{} visit-\Inf{} you.\Acc{}//
        \glft 	\enquote*{I am making an effort to visit you.}\trailingcitation{\parencite[54]{EKiss2002}}//
    \endgl
\xe

\pex\label{ex:tv-inf}
    \a
    \begingl
        \glpreamble \textbf{Transitive matrix verb, intransitive infinitive}//
        \gla 	János meg-próbál-t \nogloss{[\tss{\Inf}} bemen-ni \nogloss{]}.//
        \glb 	János \Vm-try-\Pst.\Tsg.\Sbj{} enter-\Inf{}//
        \glft 	\enquote*{János tried to go in.}\trailingcitation{\parencite[153]{EKiss1989}}//
    \endgl
    \a
    \begingl
        \glpreamble \textbf{Transitive matrix verb, transitive infinitive}//
        \gla 	Anna meg-próbál-ta \nogloss{[\tss{\Inf}} meg-tanul-ni a vers-et \nogloss{].}//
        \glb 	Anna \Vm-try-\Pst.\Tsg.\Sbj>\Third.\Obj{} \Vm-learn-\Inf{} the poem-\Acc//
        \glft 	\enquote*{Anna tried to learn the poem.}\trailingcitation{\parencite[33]{KenVagFeny1998}}//
    \endgl
\xe

It is generally argued in the literature that if the matrix verb cannot take an
\Acc{} \gls{DO}, it will always show subject agreement even if the \gls{DO} of
its infinitival complement is a definite third person object, as
in~(\ref{ex:itv-inf}a,b). The only exception to this is claimed to occur with
second person \glspl{DO}, which allow the \emph{-lak/-lek} object agreement
form, as in (\ref{ex:itv-inf}c). With matrix verbs that do take \Acc{}
\glspl{DO}, agreement on the matrix verb depends on the definiteness of the
infinitive's \gls{DO}.

The main claim of this paper is that the empirical picture is more complex than
indicated by~\eqref{ex:itv-inf} and~\eqref{ex:tv-inf}. In particular, there is
evidence that intransitive matrix verbs, that is verbs that do not by
themselves take \Acc{} \glspl{DO}, \textbf{can nevertheless agree with the
definite third person object of the infinitive}.  An example is shown
in (\nextx):\glsunset{MNSZ}

\ex \textbf{Intransitive matrix verb, transitive infinitive and object agreement}\\
    \begingl
        \glpreamble \Obj --- finite verb --- \Inf{} [321]; \gls{MNSZ}/doc\#2886//
        \gla 	\nogloss{\dots{}} hogy élet-em egyik legnagyobb hülyeség-é-t
        \textbf{készül-öm} {véghez vin-ni}.//
        \glb 	that life-\Fsg.\Poss{} one biggest
        idiocy-\Tsg.\Poss-\Acc{} get.ready bring.about-\Inf{}//
        \glft 	\enquote*{\dots{} that I am getting ready to bring about one
        of the biggest idiocies of my life.}//
    \endgl
\xe

I suggest that speakers who produce and allow structures like (\lastx) do so in
analogy to the structure in~\eqref{ex:tv-inf}. The verbs in~\eqref{ex:itv-inf}
and (\lastx) do not generally have \Acc{} \glspl{DO} and do not agree with any
non-subject argument, but the verbs in~\eqref{ex:tv-inf} have \Acc{} \glspl{DO}
of their own and agree with them, or straightforwardly agree with the
\glspl{DO} of their infinitival copmlement (hence \enquote{long-distance}
agreement).

In addition, I argue that the data shown below and found in corpora indicate
that second person \glspl{DO} of infinitival complements do not trigger object
agreement more readily than third person objects, suggesting that there is a
single agreement mechanism responsible for both. Differences in acceptability
of second vs.\ third person objects in these contexts as reported by
\textcites[227]{EKiss1987}[54]{EKiss2002}[61]{KalmanCetal1989}[451]{denDikken2004}
are weaker than expressed there or they might be due to other factors,
including verb morphology.

\section{Data}\label{sec:data}

\eqref{data:itv-inf} presents some intransitive predicates, which do not have
\Acc{} \glspl{DO}, and which are said not to agree with the \Acc{} object of
their infinitival complement (see e.g.~\citealt[226]{EKiss1987},
\citealt[54]{EKiss2002}, \citealt[60--61]{KalmanCetal1989},
\citealt[79]{SzeSze2018} on \emph{igyekszik}, \citealt[61]{KalmanCetal1989},
\citealt[449, 451]{denDikken2004} on \emph{jön}, \citealt[79]{SzeSze2018} on
and \emph{készül}).

\ex\label{data:itv-inf}\textbf{Intransitive verbs (no \Acc{} \gls{DO}) taking infinitival complements}\\
    \emph{igyekszik} \enquote*{strive}, \emph{jár} \enquote*{go (regularly)},
    \emph{(el)jön} \enquote*{come}, \emph{készül} \enquote*{prepare},
    \emph{próbálkozik} \enquote*{attempt}, \emph{siet} \enquote*{hurry},
    \dots{}
\xe
\eqref{data:tv-inf} shows transitive verbs which take \Acc{} \glspl{DO} and
which allow object agreement; whether agreement appears or not depends on
syntactic and semantic properties of the object
\parencite{Bartos1999,EKiss2002,denDikken2006,CopWec2012b,Coppock2013P,Barany2015d,Barany2017}.

\ex\label{data:tv-inf}\textbf{Transitive verbs (\Acc{} \gls{DO}) taking infinitival complements}\\
    \emph{akar} \enquote*{want}, \emph{fog} (future auxiliary),
    \emph{megpróbál} \enquote*{try}, \emph{un} \enquote*{find boring},
    \emph{utál} \enquote*{hate}, \dots{}
\xe

\subsection{Agreement of intransitive verbs with \Third{}rd person
    objects}\label{sub:agr-3rd}

The intransitive verbs in~\eqref{data:itv-inf} lacking \Acc{} \glspl{DO} can
appear with both \Sbj{} and \Obj{} agreement when they have infinitival
complements, in what seem to be the exact same environments as the transitive
verbs in~\eqref{data:tv-inf}.
%
In this section, I illustrate a selection of attested examples with the
predicates listed in~\eqref{data:itv-inf} and subjects with different
φ-features. The data are from the Hungarian National Corpus, the
\enquote{Magyar nemzeti szövegtár} (MNSZ; \url{http://corpus.nytud.hu/mnsz/})
and other sources on the internet (see~\Cref{app:sources}). Each example is
coded with a permutation of 123, indicating the order of the finite matrix verb
(1), the infinitive (2) and the object (3).

\subsubsection{First person singular subject, third person
object}\label{ssub:fsg-obj}

Clear examples of intransitive predicates that agree with a first person
singular subject, as well as the object of the infinitival complement (glossed
as \Fsg.\Sbj>\Third.\Obj) were only found for the predicate \emph{készül}
\enquote*{get ready}. This is partly for morphological reasons: the \emph{-m}
suffix is the syncretic exponent of \Fsg.\Sbj{} agreement in the past tense,
where the distinction between object agreement and its absence is neutralised,
as well as the single exponent for first person subjects (with or without
object agreement) for the class of \emph{-ik}-verbs, which have a \Third.\Sg{}
marker \emph{-ik} in place of the regular null marker. This rules out finding
relevant examples for \emph{igyekszik} and \emph{próbálkozik}, for exampe. With
\emph{készül}, I have found a total of nine examples with the form
\emph{készülöm} shown in~\eqref{ex:windows-keszul} out of a total of 30
examples with \emph{készül} (29 with third person objects).

\ex\label{ex:windows-keszul}%
    \begingl
        \glpreamble \Obj{} --- finite verb --- \Inf{} [312];
        \Cref{app:sources}//
        \gla 	A {Windows XP}-t \textbf{készül-öm} levált-ani linux-ra \nogloss{\dots}//
        \glb 	the {Windows XP}-\Acc{} prepare-\Fsg.\Sbj>\Third.\Obj{} change-\Inf{} linux-\Subl{}//
        \glft 	\enquote*{I am planning to switch from Windows XP to Linux.}//
    \endgl
\xe

\subsubsection{Second person singular subject, third person object}\label{sub:2-3}

The verb forms expressing agreement with a second person singular subject and a
third person object \emph{-od/-ed/-öd} are not syncretic in the relevant
configurations, and it is easier to find relevant examples for different
predicates, for example \emph{igyekszik}, \emph{készül}, \emph{próbálkozik},
and \emph{siet}. In~\eqref{ex:meg-probal}, the infinitive's object is
\emph{pro}, licensed by object agreement on the finite verb. In addition, the
verbal modifier \emph{meg}, selected by the infinitive \emph{nyitni}, is
spelled out in a higher position in the matrix clause, a property of some but
not all transitive verbs in~\eqref{data:tv-inf}
(see~\citealt[18--22]{EKissvanRie2004b} for discussion).

\ex\label{ex:oxford-keszul}%
    \begingl
        \glpreamble finite verb --- \Inf{} --- \Obj{} [123]; \Cref{app:sources}//
        \gla 	Bocs, ha épp \textbf{készül-t-ed} betanul-ni az Oxford nagyszótár-at.//
        \glb 	sorry if just prepare-\Pst-\Ssg.\Sbj>\Third.\Obj{} learn the Oxford big.dictionary-\Acc//
        \glft 	\enquote*{Sorry if you were just preparing to learn the Oxford dictionary by heart.}\trailingcitation{}//
    \endgl
\xe
\ex\label{ex:siet-almagyar}
    \begingl
        \glpreamble finite verb --- \Inf{} --- \Obj{} (CP) [123]; \gls{MNSZ}/doc\#972//
        \gla 	Hiszen mindig \textbf{siet-ed} {kikér-ni magad-nak}, hogy ál-magyar len-né-l.//
        \glb 	since always hurry-\Tsg.Sbj>\Tsg.\Obj{} protest that fake-Hungarian be-\Cond.\Tsg-\Sbj{}//
        \glft 	\enquote*{Since you always hurry to protest that you're a fake Hungarian.}//
    \endgl
\xe
\ex\label{ex:meg-probal}%
    \begingl
        \glpreamble finite verb --- \Inf{} --- \emph{pro} [12\emph{pro}]; \Vm{}-climbing; \Cref{app:sources}//
        \gla 	\nogloss{\dots} de most teljes üresség van, ha meg
        \textbf{próbálkoz-od} nyit-ni.//
        \glb 	but now complete emptiness \Cop{} if \Vm{} try-\Ssg.\Sbj>\Third.\Obj{} open-\Inf{}//
        \glft 	\enquote*{\dots{} but now it's completely empty if you try to open it}\trailingcitation{}//
    \endgl
\xe

\subsubsection{Third person singular subject, third person object}\label{sub:3-3}

Intransitive predicates are also attested showing agreement with a third person
singular subject and the third person object of their infinitival complement.
The following examples illustrate \emph{készül} and \emph{szándékozik}
\enquote*{to intend}. Analogous constructions with \emph{igyekszik} and
\emph{jár} are also attested in the data set.

\ex\label{ex:birtok}%
    \begingl
        \glpreamble finite verb --- \Inf{} --- \Obj{} [123]; \Cref{app:sources}//
        \gla 	\nogloss{\dots} birtok-ba \textbf{készül-i} ven-ni az új föld-jé-t.//
        \glb 	possession-\Ill{} prepare-\Tsg.\Sbj>\Third.\Obj{} take-\Inf{} the new land-\Tsg-\Acc//
        \glft 	\enquote*{\dots{} he wants to take his new plot of land into possession.}\trailingcitation{}//
    \endgl
\xe
\ex\label{ex:barat-szandek}%
    \begingl
        \glpreamble \Obj{} --- finite verb --- X --- \Inf{} [312]; \Cref{app:sources}//
        \gla 	Barát-já-t \textbf{szándékoz-t-a} magá-val vin-ni.//
        \glb 	friend-\Tsg-\Acc{} intend-\Pst-\Tsg.\Sbj>\Third.\Obj{} \Refl.\Tsg{}-\Com{} bring-\Inf{}//
        \glft 	\enquote*{S/he intended to bring his/her friend along.}//
    \endgl
\xe

\subsubsection{First person plural subject, third person object}\label{sub:1pl-3}

First person plural subjects are also found in the relevant constructions,
shown here for \emph{készül} and \emph{siet}, and also attested for
\emph{igyekszik} and \emph{szándékozik}.

\ex\label{ex:meat-market}%
    \begingl
        \glpreamble \Inf --- finite verb --- \Obj{} [213]; \Cref{app:sources}//
        \gla 	\nogloss{\dots} megválaszt-juk a ruhá-nk-at, megcsinál-juk a
        frizurá-nk-at, az internetes húspiac-on is ugyanúgy elad-ni
        \textbf{készül-jük} magunk-at.//
        \glb 	choose-\Fpl.\Sbj>\Third.\Obj{} the clothes-\Fpl-\Acc{} do-\Fpl.\Sbj>\Third.\Obj{} the hair-\Fpl-\Acc{} the internet.\Adj{} {meat market}-\Supe{} too likewise sell-\Inf{} prepare-\Fpl.\Sbj>\Third.\Obj{} \Refl.\Fpl-\Acc//
        \glft 	\enquote*{\dots{} we choose our clothes, we do our hair, and in the same way we prepare to sell ourselves on the online meat market.}\trailingcitation{}//
    \endgl
\xe
\ex\label{ex:siet-civilizacio}
    \begingl
        \glpreamble finite verb --- \Inf{} --- \Obj{} [132]; \Cref{app:sources}//
        \gla 	Egy-egy ugrás-sal \textbf{siet-t-ük} utolér-ni a civilizáció-ban és a politikai előhaladás-ban a többi európai nemzet-ek-et \nogloss{\dots}//
        \glb 	one-one jump-\Com{} hurry-\Pst-\Fpl{} {catch up}-\Inf{} the civilisation-\Ine{} and the political progress-\Ine{} the other European nation-\Pl-\Acc{}//
        \glft 	\enquote*{We hurried to catch up the other European nations in
        civilisation and political progress with one step or another \dots}//
    \endgl
\xe

\subsubsection{Second person plural subject, third person object}\label{sub:2pl-3}

The following examples have second person plural subjects. \eqref{ex:mennyi},
with \emph{készül}, again shows a \emph{pro} object. \eqref{ex:remalom}
and~\eqref{ex:magatokat} are present and past tense examples of
\emph{igyekszik}.

\ex\label{ex:mennyi}%
    \begingl
        \glpreamble finite verb --- \Inf{} --- \emph{pro}; \Cref{app:sources}//
        \gla 	Mennyi-ért \textbf{készül-itek} ven-ni?//
        \glb 	{how much}-for prepare-\Spl.\Sbj>\Third.\Obj{} buy-\Inf{}//
        \glft 	\enquote*{For how much are you preparing to buy it [a computer]?}\trailingcitation{}//
    \endgl
\xe
\ex\label{ex:remalom}%
    \begingl
        \glpreamble finite verb --- \Obj{} --- \Inf{} [132]; \Cref{app:sources}//
        \gla 	\nogloss{\dots} azon kívül, hogy \textbf{igyeksz-itek} ez-t a rémálm-ot elfelejt-eni, \nogloss{\dots}//
        \glb 	that apart that strive-\Spl.\Sbj>\Third.\Obj{} this-\Acc{} the nightmare-\Acc{} forget-\Inf//
        \glft 	\enquote*{\dots{} apart from the fact that you strive to forget this nightmare \dots{}}\trailingcitation{}//
    \endgl
\xe
\ex\label{ex:magatokat}%
    \begingl
    \glpreamble finite verb --- \Inf{} --- \Obj{} [213]; \Cref{app:sources}//
        \gla 	Mi-vel \textbf{igyekez-t-étek} megnyugtat-ni magatokat, amikor elhagyott a szerelmetek több év után?//
        \glb 	what-\Com{} strive-\Pst-\Tsg.\Sbj>\Third.\Obj{} calm-\Inf{} \Refl-\Tpl-\Acc{} when left the love-\Tpl{} several year after//
        \glft 	\enquote*{How did you try to calm yourselves when your lover left you after several years?}\trailingcitation{}//
    \endgl
\xe

\subsubsection{Third person plural subject, third person object}\label{sub:3pl-3}

Examples with third person plural subjects, agreement with third person
definite objects (\Tpl.\Obj):

\ex\label{ex:szandek-bp}
    \begingl
    \glpreamble \Obj{} --- \Inf{} --- finite verb [321]; \gls{MNSZ}/doc\#901//
        \gla 	\nogloss{\dots{}} hogy valaki-k a Fővárosi Önkormányzat-ot meg-károsít-ani \textbf{szándékoz-zák} vagy \textbf{szándékoz-t-ák} //
        \glb 	that someone-\Pl{} the capital.\Adj{} local.government-\Acc{} \Vm-harm-\Inf{} intend-\Tpl.\Sbj{}>\Third.\Obj{} or intend-\Pst-\Tpl.\Sbj{}>\Third.\Obj{}//
        \glft 	\enquote*{that some people intend or intended to harm the General Assembly of Budapest}//
    \endgl
\xe
\ex\label{ex:lovag-igyek}
    \begingl
        \glpreamble finite verb --- \Inf{} --- \Obj{} [123]; \Cref{app:sources}//
        \gla 	Ezért a német lovag-ok a 14. század-ban \textbf{igyekez-t-ék} elfoglal-ni Litvániá-t is.//
        \glb 	{because of this} the German knight-\Pl{} the 14th century-\Ine{} strive-\Pst-\Tpl.\Obj{} conquer-\Inf{} Lithuania-\Acc{} too//
        \glft 	\enquote*{Because of this, in the 14th century the German knights strove to conquer Lithuania as well.}\trailingcitation{}//
    \endgl
\xe

\section{Distribution of agreement}\label{sec:distribution-of-agreement}

\subsection{Person}

\Cref{tb:lda-distr} shows that object agreement with third person definite
objects is found with intransitive verbs, in contrast to many claims in the
literature. This is true for any combination of person of subject and object
where object agreement is overtly coded. In particular, the difference between
agreement with \Second{}nd and \Third{}rd person objects is not
categorical: both can trigger object agreement.
%
Each cell in~\Cref{tb:lda-distr} with \faCheck{} has at least one attested
instance of agreement with an object of that person. In the empty cells
in~\Cref{tb:lda-distr}, there are no distinct object agreement forms in the
first place.

\begin{table}[htpb]
    \centering
    \begin{tabular}{llll}
    \toprule
    \Sbj{} / \Obj{} & \First{} & \Second{}           & \Third{} \\
    \midrule
    \Fsg{}          &          & (\ref{ex:itv-inf}c) & \eqref{ex:windows-keszul} \\
    \Fpl{}          &          &                     & \eqref{ex:meat-market}--\eqref{ex:siet-civilizacio}\\
    \Ssg{}          &          &                     & \eqref{ex:oxford-keszul}--\eqref{ex:meg-probal}\\
    \Spl{}          &          &                     & \eqref{ex:mennyi}--\eqref{ex:magatokat} \\
    \Tsg{}          &          &                     & \eqref{ex:birtok}--\eqref{ex:barat-szandek} \\
    \Tpl{}          &          &                     & \eqref{ex:szandek-bp}--\eqref{ex:lovag-igyek} \\
    \bottomrule
    \end{tabular}
    \caption{Distribution of LDA with intransitive matrix
    verbs}\label{tb:lda-distr}
\end{table}

It is clear, however, that the overall frequency of long-distance agreement
with these predicates that do not take \Acc{} \glspl{DO} is much lower than
with transitive verbs like \emph{akar}, for
example.~\Cref{tb:mnsz-person-counts} shows the distribution of different
person combinations for \emph{igyekszik}, \emph{készül} and
\emph{szán\-dé\-kozik} as well as \emph{akar} from the \gls{MNSZ} (disregarding
examples from other sources). For the intransitive verbs, these are the total
number of occurrences in both present and past tense, while for \emph{akar} I
randomly sampled 500 occurrences for each tense, with 371 present and
416 past occurrences remaining after removing misclassified examples and
duplicates. \Fsg>\Third{} is not taken into account because of syncretism of
the relevant forms.

\begin{table}[htpb]
    \centering
    \begin{tabular}{lllllll}
    \toprule
        & \Fpl{}>\Third{} & \Fsg{}>\Second{} & \Ssg{}>\Third{} & \Spl{}>\Third{} & \Tsg{}>\Third{} & \Tpl{}>\Third{} \\
    \midrule
    \emph{igyekszik} & 84 & 17 & 13 & 0 & 84 & 148\\
    \emph{készül} & 2 & 0 & 0 & 0 & 3 & 5 \\
    \emph{szándékozik} & 43 & 0 & 8 & 1 & 156 & 173\\
    \midrule
    \emph{akar} & 50 & 13 & 31 & 3 & 437 & 253\\
    \bottomrule
    \end{tabular}
    \caption{Distribution of person configurations for different verbs
    from the \gls{MNSZ}}\label{tb:mnsz-person-counts}
\end{table}

While the totals for each row differ strongly, the distribution of person forms
in each row is relatively similar. Third person subjects are the most frequent
for each verb. While this is probably partly due the nature of the texts in the
corpus, it is worth noting that \Fpl>\Third{} forms are more frequent than
\Fsg{}>\Second{} for each verb as well, even though \Fsg>\Second{} has been
claimed to be the only grammatical form of long-distance agreement for
intransitive verbs such as \emph{igyekszik}.

\subsection{Word orders}\label{sub:word-orders}

The examples in~\Cref{sec:data} show five of the six possible permutations of
the order of the finite matrix verb (1), the infinitive (2) and the object (3),
shown in~\Cref{tb:data-orders}.

\begin{table}[htpb]
    \centering
    \begin{tabularx}{1\textwidth}{lXXXXXX}
    \toprule
        & 123 & 132 & 213 & 231 & 312 & 321 \\
    \midrule
    Ex. & \eqref{ex:oxford-keszul},~\eqref{ex:birtok},~\eqref{ex:lovag-igyek} &
                \eqref{ex:remalom} &
                    \eqref{ex:meat-market},~\eqref{ex:magatokat} &
                        &
                            \eqref{ex:windows-keszul},~\eqref{ex:barat-szandek} &
                                \eqref{ex:szandek-bp} \\
    \bottomrule
    \end{tabularx}
    \caption{Distribution of word orders in examples
    from~\Cref{sec:data}}\label{tb:data-orders}
\end{table}

Orders 312 and 213 indicate movement of either the object (312) or the
infinitive (213) into the matrix clause, often as a focus. Both of these orders
lead to adjacency between the finite verb and the object in the majority of
cases\footnote{Maybe adjacency helps construing the object as an argument of
the matrix verb \parencite{Peredy2009}.} but object agreement is found without
adjacency as well. Order 231 involves fronting both the infinitive, as a
(contrastive) topic, and the object, as a matrix focus; an attested example
with the transitive verb \emph{akar} is shown in (\nextx):

\ex
    \begingl
    \glpreamble \Inf{} --- \Obj{} --- finite verb [231]; \gls{MNSZ}/doc\#2201//
        \gla 	\nogloss{\dots{}} de legyűr-ni ők-et valójában nem akar-t-ák.//
        \glb 	but overcome-\Inf{} \Tpl-\Acc{} really not want-\Pst-\Tpl.\Sbj{}>\Third.\Obj{}//
        \glft 	\enquote*{\dots{} but they really did not want to overcome them.}//
    \endgl
\xe
I do not see a principled reason for ruling out 231 (as in (\lastx)) with an
intransitive verb like \emph{igyekszik}, \emph{készül}, etc., given the range
of data found with other orders shown in~\Cref{tb:data-orders}.
%
However, among the 787 examples of \emph{akar} with infinitival complements,
(\lastx) was the only example with 231 order, suggesting that this order is
generally rare, not just when the matrix verb is intransitive.

\Cref{tb:mnsz-order-counts} shows the distribution of word orders for the four
verbs from~\Cref{tb:mnsz-person-counts} with their proportions. Once again, the
total numbers considerably differ for the intransitive verbs vs.\ \emph{akar},
but the distributions are similar: 123 is the most common order for
\emph{igyekszik}, \emph{szándékozik} and \emph{akar}, with 312 the second most
frequent.

\begin{table}[htpb]
    \centering
    \begin{tabular}{llllllll}
    \toprule
        & 123 & 132 & 213 & 231 & 312 & 321 & \emph{pro} \\
    \midrule
    \emph{igyekszik} & 174 & 36 & 5 & 0 & 93 & 6 & 32\\
    \emph{készül} & 3 & 0 & 6 & 0 & 1 & 0 & 2 \\
    \emph{szándékozik} & 102 & 22 & 35 & 0 & 161 & 30 & 31 \\
    \midrule
    \emph{akar} & 361 & 50 & 35 & 1 & 221 & 7 & 112 \\
    \bottomrule
    \end{tabular}
    \caption{Distribution of person configurations for different verbs
    from the \gls{MNSZ}}\label{tb:mnsz-order-counts}
\end{table}

In sum, five word orders are attested for \emph{akar} with infinitival
complements as well as for \emph{igyekszik} and \emph{szándékozik} (see also
the data in~\Cref{sec:data}), with 231 being the only exception. Orders might
be influenced by information structure, exhibiting focus and topic movement,
but there do not seem to be any clear differences in the distribution of word
orders for transitive or intransitive matrix verbs.

\subsection{Past tense}\label{sub:past-tense}

\Textcite{denDikken2004} points out that the grammaticality of object
agreement, in particular \Second{}nd person agreement, depends on tense with
verbs forming \enquote{come/go verb aspectual constructions} . For example,
\emph{jön} can form a \Fsg>\Second.\Obj{} form in the past but not the present
tense, as shown in (\nextx).

\ex
    \begingl
        \gla 	Jö-tt-elek \nogloss{/ } \ljudge*jö-lek meg-látogat-ni \nogloss{(} @ téged\emph{)}.//
        \glb 	come-\Pst-\Fsg>\Second.\Obj{} come-\Fsg>\Second.\Obj{} \Vm-visit-\Inf{} you.\Acc{}//
        \glft 	\enquote*{I came to visit you.}\trailingcitation{\parencite[451]{denDikken2004}}//
    \endgl
\xe
Other verbs with similar semantics and argument structure, like \emph{jár}
\enquote*{go (regularly)} can form \Fsg>\Second.\Obj{} in both present and
past, although as with all data presented here, there is variation in how
acceptable these forms are:

\ex
    \begingl
        \gla 	Jár-lak \nogloss{/} jár-ta-lak meg-látogat-ni \nogloss{(}téged\emph{)}.//
        \glb 	go-\Fsg>\Second.\Obj{} go-\Pst-\Fsg>\Second.\Obj{} \Vm-visit-\Inf{} you.\Acc//
        \glft 	\enquote*{I go to visit you regularly.}//
    \endgl
\xe

A reason for why past tense forms like \emph{jö-tt-elek}
\enquote*{come-\Pst-\Fsg>\Second.\Obj} are more acceptable than their present
tense counterparts \emph{*jö(l)-lek} \enquote*{come-\Fsg>\Second.\Obj} can lie
in morphology. The present tense forms of \emph{jön}, \emph{megy}, \emph{lenni}
are irregular, while their past tense forms are regular, based on a single stem
ending in \emph{-t}. It is straightforward to form analogical (agreeing)
patterns based on transitive forms in the past; this is not possible in the
present tense --- cf.~\Cref{tb:jon-prs-pst}.

\begin{table}[h!]%[htpb]
    \centering
    \begin{tabular}{lllll}
    \toprule
              & Present        & Past              & Present       & Past \\
    \midrule
    \Fsg{} & \emph{jöv-ök}  & \emph{jö-tt-em}   & \emph{jár-ok}  & \emph{jár-t-am} \\
    \Ssg{} & \emph{jö-sz}   & \emph{jö-tt-él}   & \emph{jár-sz}  & \emph{jár-t-ál} \\
    \Tsg{} & \emph{jön}     & \emph{jö-tt}      & \emph{jár}     & \emph{jár-t} \\
    \Fpl{} & \emph{jöv-ünk} & \emph{jö-tt-ünk}  & \emph{jár-unk} & \emph{jár-t-unk} \\
    \Spl{} & \emph{jöt-tök} & \emph{jö-tt-etek} & \emph{jár-tok} & \emph{jár-t-atok} \\
    \Tpl{} & \emph{jön-nek} & \emph{jö-tt-ek}   & \emph{jár-nak} & \emph{jár-t-ak} \\
    \bottomrule
    \end{tabular}
    \caption{Present and past tense forms of \emph{jön} \enquote*{come}
    (irregular) and \emph{jár} \enquote*{go (regularly)}}\label{tb:jon-prs-pst}
\end{table}

As \textcite{denDikken2004} also mentions, \emph{jár}, while generally
intransitive, can be used transitively with locational objects
straightforwardly (also with different \Vm{}s), for example in \emph{jár-ja az
útját} \enquote*{I am going my way}.
%
In contrast to the predicates in~\Cref{sec:data}, however, it agrees with the
object of the infinitive in even fewer cases. An attested example is shown in
(\nextx).

\ex\label{ex:jarta-nezni}
    \begingl
        \glpreamble finite verb --- \Inf{} --- \Obj{} [123]; \Cref{app:sources}//
        \gla 	Két nap-ig a falu nép-e jár-t-a néz-ni a fölakasztott ember-t.//
        \glb 	two day-\Term{} the village people-\Tsg{} go-\Pst-\Tsg.\Sbj>\Third.\Obj{} watch-\Inf{} the hung person-\Acc{}//
        \glft 	\enquote*{The villagers went to watch the hung person for two days.}//
    \endgl
\xe

It is not clear what causes different frequencies of long-distance agreement in
the present and past tense, although morphological regularity arguably plays a
role. For \emph{igyekszik}, \emph{szándékozik} and \emph{akar}, past tense
forms are more frequent in the \gls{MNSZ}, although this is probably again
influenced by the nature of the texts in the corpus. The ratios of past tense
to present are roughly equal for \emph{szándékozik} (1.14) and \emph{akar}
(1.18) but higher for \emph{igyekszik} (6.69). This could be related to
morphology as well: \emph{igyekszik} has less regular present tense forms
than the other verbs.

\subsection{Summary}\label{sub:summary}

The main difference between typical long-distance agreement with a transitive
verb like \emph{akar} and the intransitive verbs surveyed in~\Cref{sec:data} is
in the overall frequency of the constructions. The total occurrences of
\emph{igyekszik} (346) and \emph{szándékozik} (381) with long-distance
agreement in the \gls{MNSZ} are a fraction of the total for \emph{akar}. The
distribution of long-distance agreement with respect to the person of the
object (and the subjcet) as well as word orders does not seem to differ
strongly for different verbs
(see~\Cref{tb:mnsz-person-counts,tb:mnsz-order-counts}).
%
The fact that a range of verbs that do not take \Acc{} objects appear in a
long-distance agreement construction in the \gls{MNSZ} and other sources on the
internet clearly indicates that there is no general ban on object agreement
with these verbs.

\section{Towards an analysis}\label{sec:analysis}

The data in the previous sections showed that agreement between an intransitive
matrix verb and the infinitive's object, albeit much less frequent than with
transitive verbs, is nevertheless \emph{regular}, i.e.\ a definite second or
third person object can trigger object agreement. The attested patterns are
schematically shown in~\eqref{ex:types}.

\pex\label{ex:types}
    \a 	{}[ V-\Sbj{} \dots{} [\tss{\Inf} V DP-\Def{} ]]\trailingcitation{\faCheck}
    \a 	{}[ V-\Obj{} \dots{} [\tss{\Inf} V DP-\Def{} ]]\trailingcitation{\faCheck}
    \a 	{}[ V-\Sbj{} \dots{} [\tss{\Inf} V DP-\Indef{} ]]\trailingcitation{\faCheck}
    \a 	{}[ V-\Obj{} \dots{} [\tss{\Inf} V DP-\Indef{} ]]\trailingcitation{\faTimes}
\xe
Transitive matrix verbs show types (\ref{ex:types}b,c). Intransitive verbs can
additionally show type (\ref{ex:types}a). But neither class would show
(\ref{ex:types}d), e.g.\ object agreement with an indefinite object.
%
These patterns can be accounted for as follows. Verbs that do not take \Acc{}
\glspl{DO} can be unergative (\emph{igyekszik}, \emph{siet}) or unaccusative
(\emph{jön}), suggesting that this property alone does not determine their
behaviour with respect to object agreement. I assume that even if unergative
verbs project a \emph{v}P that introduces their agent, this \emph{v} might lack
a φ-probe that can agree with an \Acc{} object.

When these verbs appear with infinitival complements, however, speakers can
treat them as analogous to transitive verbs like \emph{akar}, \emph{fog}, or
\emph{un} which regularly agree with the infinitive's \gls{DO} by construing
the structures as having a φ-probe on \emph{v}, independently of the type of
matrix verb, allowing it to agree with the infinitive's object.

\pex[interpartskip=6ex]\label{ex:derivations}
    \a  Agreement with a definite third person object\\
        \dots{} [\tss{\emph{v}P} \tn{v}{\emph{v}[uφ]} [\tss{VP} [\tss{XP} V.\Inf{} \tn{dp}{DP.\Def{}} ]]]
        \td{v}{dp}{\faCheck{} Agree}
    \a  No agreement with an indefinite third person object\\
        \dots{} [\tss{\emph{v}P} \tn{v}{\emph{v}[uφ]} [\tss{VP} [\tss{XP} V.\Inf{} \tn{dp}{DP.\Indef{}} ]]]
        \tx{v}{dp}{\faTimes{} Agree}
    \a  No probe on intransitive \emph{v}\\
        \dots{} [\tss{\emph{v}P} \tn{v}{\emph{v}} [\tss{VP} [\tss{XP} V.\Inf{} \tn{dp}{DP.\Def/\Indef{}} ]]]
\xe
Transitive verbs only allow (\ref{ex:derivations}a,b), but not
(\ref{ex:derivations}c). Intransitive verbs vary: speakers who treat them
analogously to transitives allow (\ref{ex:derivations}a,b), others allow
(\ref{ex:derivations}c), the \enquote{standard} case. But there is no way to
derive a pattern like~(\ref{ex:types}d).

\section{Conclusions}\label{sec:conclusions}

I have argued that verbs that do not take \Acc{} objects can nevertheless show
object agreement with definite \Acc{} objects of their infinitival complements.
While this possibility has generally only been acknowledged for second person
objects (and first person subjects) in the literature, the data shown here
indicate that the distribution of long-distance agreement with intransitive
verbs is in fact much wider.
%
This suggests that object agreement with second and third person objects should
not be analysed as categorically different, as both are robustly attested in
long-distance agreement involving intransitive verbs.
%
It is also clear that while attested, long-distance agreement involving
intransitive verbs is much rarer than with transitive verbs. This difference
could be due to grammatical pressure against using verbs without \Acc{} objects
with object agreement morphology. On the other hand, analogy to transitive
verbs in the same contexts can license agreement for some speakers.

\section*{Acknowledgments}

This paper is dedicated to Katalin É.\ Kiss, who has been an extremely helpful
and motivating colleague over the past ten years and whose linguistic work I
have always admired.

In addition, I want to thank Krisztina Szécsényi for discussing this topic with
me several times, Júlia Keresztes and Ádám Szalontai (as well as a few other
native speakers) for judgements, and Anna Bruggeman and Ádám Szalontai for
discussion of the quantitative aspects of this paper.

\newrefcontext[sorting=nyt]
\printbibliography

\appendix

\section{Sources}\label{app:sources}

See \url{http://github.com/andrasbarany/hungarian-lda/} for the full data set
which also includes the search terms used in the \gls{MNSZ} as well as their
document identifiers, and URLs for the data from other sources on the internet.

{\sloppy
{\scriptsize
\begin{itemize}

    \item[\eqref{ex:windows-keszul}] \url{https://sg.hu/forum/tema/986486185}
    \item[\eqref{ex:oxford-keszul}] \url{http://www.angolnyelvtanitas.hu/angolnyelvtanitas-blog/tevhit-gyilkos-nyelvtanulasi-ertelmezo-kisszotar}
    \item[\eqref{ex:meg-probal}] \url{https://www.fanfic.hu/merengo/reviews.php?sid=117490&a=1}
    \item[\eqref{ex:birtok}] \url{http://spareoom.uw.hu/picspam/Rome_1x12.htm}
    \item[\eqref{ex:barat-szandek}]
        \url{https://library.hungaricana.hu/hu/view/BacsKiskunMegyeiNepujsag_1977_01/?pg=156\&layout=s}
    \item[\eqref{ex:meat-market}] \url{http://lelkikoto.blog.hu/2017/03/16/letoltott_szerelem}
    \item[\eqref{ex:siet-civilizacio}]
        \url{https://www.arcanum.hu/hu/online-kiadvanyok/Mikszath-mikszath-osszes-muve-2A85B/cikkek-es-karcolatok-5186-kotet-40CD5/1880-szegedi-naplo-vezercikkek-es-egyeb-politikai-cikkek-59-kotet-4578F/243-sz-oktober-16-a-fenyuzes-45962/}
    \item[\eqref{ex:mennyi}] \url{https://www.gyakorikerdesek.hu/szamitastechnika__hardverek__4002418-vailant-notebook-ot-fogunk-holnap-venni-anyukammal1ghz-processzor256-ram20-gb}
    \item[\eqref{ex:remalom}] \url{http://www.varoszoba.hu/2016/03/12/mennyire-eredmenyes-a-harmas-kezeles/}
    \item[\eqref{ex:magatokat}] \url{https://www.gyakorikerdesek.hu/csaladi-kapcsolatok__egyeb-kerdesek__1380679-mivel-igyekeztetek-megnyugtatni-magatokat-amikor-elhagyott-a-szerelmetek-tobb-e}
    \item[\eqref{ex:lovag-igyek}] \url{https://hu.wikipedia.org/wiki/Szamogitia}
    \item[\eqref{ex:jarta-nezni}] \url{http://docplayer.hu/8200672-Az-eucharisztia-elott.html}

\end{itemize}}}


\end{document}
