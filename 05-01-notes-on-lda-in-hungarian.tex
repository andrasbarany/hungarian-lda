% arara: xelatex: { options: [ '-no-pdf' ] }
% arara: biber if found('log', 'run Biber on the file')
% arara: makeglossaries if changed('glo')
% arara: xelatex: { options: [ '-no-pdf' ] } if found('log', 'Please rerun')
% arara: xelatex
\input{\string~/Cloud/LaTeX/Templates/article-preamble.tex}

\addbibresource{legenda.bib}
\usepackage{enumitem}

\newleipzig{MNSZ}{MNSZ}{Magyar Nemzeti Szövegtár (Hungarian National Corpus)}

\title{Notes on long-distance agreement in Hungarian}

\begin{document}

%\maketitle

\thispagestyle{plain}
\noindent{\Large\textsf{Notes on long-distance agreement in Hungarian}}\\[.2cm]
\noindent\textsf{András Bárány}

\section{Introduction}\label{sec:introduction}

Infinitival complements in Hungarian appear with different classes of verbs. It
is usually said that transitive verbs taking infinitival complements can agree
with the object of the infinitive, but intransitives cannot
\parencite{EKiss1987,EKiss1989,KalmanCetal1989,KenVagFeny1998,EKiss2002,EKissvanRie2004b,denDikken2004,Coppock2012d,Szecsenyi2017,SzeSze2018},
where transitive verbs are those that can take an accusative object and
intransitive verbs are those that cannot.

The construction in question is shown in~\eqref{ex:inf-compl}, with examples
in~\eqref{ex:itv-inf} and~\eqref{ex:tv-inf}.\footnote{Abbreviations:
\printglosses}

\ex\label{ex:inf-compl}\textbf{Matrix verb with infinitival complement}\\
    {}[ \dots{} finite verb [\tss{\Inf} infinitive (object-\Acc{}) ]]
\xe

\pex\label{ex:itv-inf}
    \a
    \begingl
        \glpreamble \textbf{Intransitive matrix verb, intransitive infinitive}//
        \gla 	János igyekez-ett \nogloss{[\tss{\Inf}} bemen-ni \nogloss{]}.//
        \glb 	János strive-\Tsg.\Pst{} enter-\Inf{}//
        \glft 	\enquote*{János strove to enter.}//
    \endgl
    \a
    \begingl
        \glpreamble \textbf{Intransitive matrix verb, transitive infinitive}//
        \gla 	Anna igyekez-ett \nogloss{[\tss{\Inf}} meg-tanul-ni a vers-et \nogloss{]}.//
        \glb 	Anna strive-\Tsg.\Pst{} \Vm-learn-\Inf{} the poem-\Acc{}//
        \glft 	\enquote*{Anna strove to learn the poem.}\trailingcitation{\parencite[33]{KenVagFeny1998}}//
    \endgl
    \a
    \begingl
        \glpreamble \textbf{Intransitive matrix verb, transitive infinitive}//
        \gla 	Igyekez-lek meglátogat-ni (téged).//
        \glb 	{make effort}-\Fsg.\Sbj>\Second.\Obj{} visit-\Inf{} you.\Acc{}//
        \glft 	\enquote*{I am making an effort to visit you.}\trailingcitation{\parencite[54]{EKiss2002}}//
    \endgl
\xe

\pex\label{ex:tv-inf}
    \a
    \begingl
        \glpreamble \textbf{Transitive matrix verb, intransitive infinitive}//
        \gla 	János meg-próbál-t \nogloss{[\tss{\Inf}} bemen-ni \nogloss{]}.//
        \glb 	János \Vm-try-\Pst.\Tsg.\Sbj{} enter-\Inf{}//
        \glft 	\enquote*{János tried to go in.}\trailingcitation{\parencite[153]{EKiss1989}}//
    \endgl
    \a
    \begingl
        \glpreamble \textbf{Transitive matrix verb, transitive infinitive}//
        \gla 	Anna meg-próbál-ta \nogloss{[\tss{\Inf}} meg-tanul-ni a vers-et \nogloss{].}//
        \glb 	Anna \Vm-try-\Pst.\Tsg.\Sbj>\Third.\Obj{} \Vm-learn-\Inf{} the poem-\Acc//
        \glft 	\enquote*{Anna tried to learn the poem.}\trailingcitation{\parencite[33]{KenVagFeny1998}}//
    \endgl
\xe
\ref{pattern:itv} and~\ref{pattern:tv} show the main patterns that are
discussed in the literature:

\begin{enumerate}[label=\textbf{\Alph*.}]

    \item 	intransitive matrix verb\label{pattern:itv}

        \begin{enumerate}[label=\arabic*.]

        \item 	\makebox[0pt][l]{intransitive infinitive}\phantom{transitive
            infinitive, \Indef{}/\Def.\Obj{}} \ding{224} finite verb with \Sbj{}
            agreement\label{itv-itv}

        \item 	\makebox[0pt][l]{transitive infinitive,
            \Indef{}/\Def.\Obj{}}\phantom{transitive infinitive,
            \Indef{}/\Def.\Obj{}} \ding{224} finite
            verb with \Sbj{} agreement\label{itv-tv}

        \item 	\makebox[0pt][l]{transitive infinitive,
            \Second{}.\Obj{}}\phantom{transitive infinitive, \Indef{}/\Def.\Obj{}}
            \ding{224} finite verb with \emph{-lak/-lek} agreement\label{itv-2}

    \end{enumerate}

    \item 	transitive matrix verb\label{pattern:tv}

    \begin{enumerate}[label=\arabic*.]

        \item 	\makebox[0pt][l]{intransitive infinitive}\phantom{transitive
            infinitive, \Indef{}/\Def.\Obj{}} \ding{224} finite verb with \Sbj{}
            agreement\label{tv-itv}

        \item 	\makebox[0pt][l]{transitive infinitive,
            \Indef{}.\Obj{}}\phantom{transitive infinitive,
            \Indef{}/\Def.\Obj{}} \ding{224} finite verb with \Sbj{}
            agreement\label{tv-indef}

        \item 	\makebox[0pt][l]{transitive infinitive,
            \Def{}.\Obj{}}\phantom{transitive infinitive, \Indef{}/\Def.\Obj{}}
            \ding{224} finite verb with \Obj{} agreement\label{tv-def}

    \end{enumerate}

\end{enumerate}

Examples~(\ref{ex:itv-inf}a--c) and (\ref{ex:tv-inf}a--b) correspond to points
\ref{pattern:itv}\ref{itv-itv}--\ref{itv-2} and
\ref{pattern:tv}\ref{tv-itv}/\ref{tv-def}, respectively. The main claim of this
paper is that the empirical picture is more complex than illustrated by these
examples. In particular, there is evidence that intransitive matrix verbs, that
is verbs that do not by themselves take \Acc{} objects, \textbf{can
nevertheless agree with the definite third person object of the infinitive}.
An example is shown in:\glsunset{MNSZ}

\ex \textbf{Intransitive matrix verb, transitive infinitive and object agreement}\\
    \begingl
        \glpreamble \gls{MNSZ}/doc\#2886//
        \gla 	\nogloss{\dots{}} hogy élet-em egyik legnagyobb hülyeség-é-t
        \textbf{készül-öm} {véghez vin-ni}.//
        \glb 	that life-\Fsg.\Poss{} one biggest
        idiocy-\Tsg.\Poss-\Acc{} get.ready bring.about-\Inf{}//
        \glft 	\enquote*{\dots{} that I am getting ready to bring about one
        of the biggest idiocies of my life.}//
    \endgl
\xe

I suggest that speakers who produce and allow patterns like (\lastx) do so in
analogy to the pattern in~\eqref{ex:tv-inf}. The verbs in~\eqref{ex:itv-inf}
and (\lastx) do not generally have \Acc{} \glspl{DO} and do not agree with any
non-subject argument, but the verbs in~\eqref{ex:tv-inf} have \Acc{} \glspl{DO}
of their own and agree with them, or straightforwardly agree with the
\glspl{DO} of their infinitival copmlement (hence \enquote{long-distance}
agreement). I sketch an analysis of this mechanism in~\Cref{sec:analysis}.

In addition, I argue that the data shown below and found in corpora indicate
that second person \glspl{DO} of infinitival complements do not trigger object
agreement more readily than third person objects, suggesting that there is a
single agreement mechanism responsible for both. Differences in acceptability
of second vs.\ third person objects in these contexts as reported by
\textcites[227]{EKiss1987}[54]{EKiss2002}[61]{KalmanCetal1989}[451]{denDikken2004}
might either be less strong than expressed there or due to other factors,
including verb morphology.

\section{Data}\label{sec:data}

\eqref{data:itv-inf} presents some intransitive predicates, which do not have
\Acc{} \glspl{DO}, and which are said not to agree with the \Acc{} object of
their infinitival complement (pattern~\ref{pattern:itv}\ref{itv-tv}) (see
e.g.~\citealt[226]{EKiss1987}, \citealt[54]{EKiss2002},
\citealt[60--61]{KalmanCetal1989}, \citealt[79]{SzeSze2018} on
\emph{igyekszik}, \citealt[61]{KalmanCetal1989}, \citealt[449,
451]{denDikken2004} on \emph{jön}, \citealt[79]{SzeSze2018} on and
\emph{készül}).

\ex\label{data:itv-inf}\textbf{Intransitive (no \Acc{} \gls{DO}) verbs taking infinitival complements}\\
    \emph{igyekszik} \enquote*{strive}\\
    \emph{jár} \enquote*{go (regularly)}\\
    \emph{(el)jön} \enquote*{come}\\
    \emph{készül} \enquote*{prepare}\\
    \emph{próbálkozik} \enquote*{attempt}\\
    \emph{siet} \enquote*{hurry}\\
\xe
\eqref{data:tv-inf} shows transitive verbs which take \Acc{} \glspl{DO} and
which allow object agreement; whether agreement appears or not depends on
syntactic and semantic properties of the object
\parencite{Bartos1999,EKiss2002,denDikken2006,CopWec2012b,Coppock2013P,Barany2015d,Barany2017}.

\ex\label{data:tv-inf}\textbf{Transitive (\Acc{} \gls{DO}) verbs taking infinitival complements}\\
    \emph{akar} \enquote*{want}\\
    \emph{fog} (future auxiliary)\\
    \emph{megpróbál} \enquote*{try}\\
    \emph{un} \enquote*{find boring}\\
    \emph{utál} \enquote*{hate}, \dots{}
\xe
%Some properties other than object agreement cross-cut the distinction between
%(\blastx) and (\lastx), e.g.\ movement of the verbal modifier (\Vm) or pre-verb
%\parencite{EKissvanRie2004b}.
%
%\pex
%    \a
%    \begingl
%        \gla 	János utál-ja Mari-t fel hív-ni telefon-on.//
%        \glb 	János hate-\Tsg.\Sbj>\Third.\Obj{} Mari-\Acc{} \Vm{} call-\Inf{} phone-\Supe{}//
%        \glft 	\enquote*{János hates to call Mary on the phone.}//
%    \endgl
%    \a
%    \begingl
%        \gla 	János fel fog-ja akar-ni hív-ni Mari-t telefon-on.//
%        \glb 	János \Vm{} \Fut.\Aux{}-\Tsg.\Sbj>\Third.\Obj{} want-\Inf{} call-\Inf{} Mari-\Acc{} phone-\Supe{}//
%        \glft 	\enquote*{János will want to call Mari on the phone.}\trailingcitation{\parencite[14]{EKissvanRie2004b}}//
%    \endgl
%\xe

\subsection{Agreement of intransitive verbs with \Third{}rd person
    objects}\label{sub:agr-3rd}

The intransitive verbs in~\eqref{data:itv-inf} lacking \Acc{} \glspl{DO} can
appear with both \Sbj{} and \Obj{} agreement when they have infinitival
complements, in what seem to be the exact same environments as the transitive
verbs in~\eqref{data:tv-inf}.

In this section, I illustrate a selection of attested examples with the
predicates listed in~\eqref{data:itv-inf} and subjects with different
φ-features. The data are from the Hungarian National Corpus, the
\enquote{Magyar nemzeti szövegtár} (MNSZ; \url{http://corpus.nytud.hu/mnsz/})
and other sources on the internet. See~\Cref{app:sources} for a link to the
full data set. Each example is coded with a permutation of [123], indicating
the order of the finite matrix verb (1), the infinitive (2) and the object (3).

\subsubsection{First person singular subject, third person
object}\label{ssub:fsg-obj}

Clear examples of intransitive predicates that agree with a first person
singular subject, as well as the object of the infinitival complement (glossed
as \Fsg.\Sbj>\Third.\Obj) were only found for the predicate \emph{készül}
\enquote*{get ready}. This is partly for morphological reasons: the \emph{-m}
suffix is the syncretic exponent of \Fsg.\Sbj{} agreement in the past tense,
where the distinction between object agreement and its absence is neutralised,
as well as the single exponent for first person subjects (with or without
object agreement) for the class of \emph{-ik}-verbs, which have a \Third.\Sg{}
marker \emph{-ik} in place of the regular null marker. This rules out finding
relevant examples for \emph{igyekszik} and \emph{próbálkozik}, for exampe. With
\emph{készül}, I have found a total of nine examples with the form
\emph{készülöm} shown in~\eqref{ex:windows-keszul} out of a total of 30
examples with \emph{készül} (29 with third person objects).

\ex\label{ex:windows-keszul}%
    \begingl
        \glpreamble \Obj{} --- finite verb --- \Inf{} [312]//
        \gla 	A {Windows XP}-t \textbf{készül-öm} levált-ani linux-ra \nogloss{\dots}//
        \glb 	the {Windows XP}-\Acc{} prepare-\Fsg.\Sbj>\Third.\Obj{} change-\Inf{} linux-\Subl{}//
        \glft 	\enquote*{I am planning to switch from Windows XP to Linux.}//
    \endgl
\xe
% \ex
%     \begingl
%         \glpreamble \Obj{} --- finite verb --- \Inf{}//
%         \gla 	\nogloss{\dots} van egy nagyon szimpatikus kölyök, most ő-t készül-\textbf{öm} elhoz-ni.//
%         \glb 	is a very nice cub now \Tsg-\Acc{} prepare-\Fsg.\Sbj>\Third.\Obj{} {pick up}-\Inf{}//
%         \glft 	\enquote*{\dots{} there's a very nice cub, I'm preparing to pick it up now.}\trailingcitation{\url{https://goo.gl/ytQ6pI}}//
%     \endgl
% \xe
%
%\ex
%    \begingl
%        \glpreamble \Obj{} --- finite verb --- \Inf{}//
%        \gla 	\nogloss{\dots} mégis most az-t készül-\textbf{öm} ten-ni.//
%        \glb 	nevertheless now that-\Acc{} prepare-\Fsg.\Sbj>\Third.\Obj{} do-\Inf//
%        \glft 	\enquote*{\dots{} nevertheless I am preparing to do that now.}\trailingcitation{\url{https://goo.gl/11iQUx}}//
%    \endgl
%\xe
%\ex\label{ex:hatnap-keszul}%
%    \begingl
%        \glpreamble \Obj{} --- \Inf{} --- finite verb [321]//
%        \gla	Most, hogy hat nap eltelté-vel beszámoló-m-at külde-ni készül-öm, \nogloss{\dots}//
%        \glb	now that six days passing-\Com{} report-\Fsg-\Acc{} send-\Inf{} prepare-\Fsg.\Sbj>\Third.\Obj{}//
%        \glft	\enquote*{Now that I am preparing to send my report after six days have passed \dots{}}\trailingcitation{}//
%    \endgl
%\xe
%\ex
%    \begingl
%        \glpreamble \Obj{} --- finite verb --- \Inf{}//
%        \gla	egyik-nél már a második falu-t készül-öm létrehoz-ni//
%        \glb	one-\Ade{} already the second village-\Acc{} prepare-\Fsg.\Sbj>\Third.\Obj{} create-\Inf{}//
%        \glft	\enquote*{With one of them, I'm preparing to create the second village already.}\trailingcitation{\url{http://forum.travian.hu/showthread.php?t=231&page=558}}//
%    \endgl
%\xe
%\ex\label{ex:bojt-keszul}%
%    \begingl
%        \glpreamble finite verb --- \Obj{} --- \Inf{} [132]//
%        \gla 	\nogloss{\dots} azért, mert készül-öm a böjt-öt fogad-ni.//
%        \glb 	{for the reason} because prepare-\Fsg.\Sbj>\Third.\Obj{} the fast-\Acc{} welcome//
%        \glft 	\enquote*{\dots{} because I am preparing to follow the fast}\trailingcitation{}//
%    \endgl
%\xe
%\ex\label{ex:elhagyni-keszul}%
%    \begingl
%        \glpreamble \Inf{} --- finite verb --- \Obj{} [213]//
%        \gla 	Elhagy-ni készül-öm a város-t.//
%        \glb 	leave-\Inf{} prepare-\Fsg.\Sbj>\Third.\Obj{} the city-\Acc{}//
%        \glft 	\enquote*{I am preparing to leave the city.}\trailingcitation{}//
%    \endgl
%\xe

\subsubsection{Second person singular subject, third person object}\label{sub:2-3}

The verb forms expressing agreement with a second person singular subject and a
third person object \emph{-od/-ed/-öd} are not syncretic in the relevant
configurations, and it is easier to find relevant examples for different
predicates, for example \emph{igyekszik}, \emph{készül}, \emph{próbálkozik},
and \emph{siet}. In~\eqref{ex:meg-probal}, the infinitive's object is
\emph{pro}, licensed by object agreement on the finite verb. In addition, the
verbal modifier \emph{meg}, selected by the infinitive \emph{nyitni}, is
spelled out in a higher position in the matrix clause, a property of some but
not all transitive verbs in~\eqref{data:tv-inf}
(see~\citealt[18--22]{EKissvanRie2004b} for discussion).

\ex\label{ex:oxford-keszul}%
    \begingl
        \glpreamble finite verb --- \Inf{} --- \Obj{} [123]//
        \gla 	Bocs, ha épp \textbf{készül-t-ed} betanul-ni az Oxford nagyszótár-at.//
        \glb 	sorry if just prepare-\Pst-\Ssg.\Sbj>\Third.\Obj{} learn the Oxford big.dictionary-\Acc//
        \glft 	\enquote*{Sorry if you were just preparing to learn the Oxford dictionary by heart.}\trailingcitation{}//
    \endgl
\xe
%\ex\label{ex:fuszer-probal}%
%    \begingl
%        \glpreamble finite verb --- \Inf{} --- \Obj{} [123]//
%        \gla 	Ha te magad próbálkoz-od összeöntöget-ni a fűszer-ek-et \nogloss{\dots}//
%        \glb 	if you yourself try-\Tsg.\Sbj>\Third.\Obj{} mix-\Inf{} the spice-\Pl-\Acc{}//
%        \glft 	\enquote*{If you try to mix the spices yourself \dots}\trailingcitation{}//
%    \endgl
%\xe
\ex\label{ex:siet-almagyar}
    \begingl
        \glpreamble finite verb --- \Inf{} --- \Obj{} (CP) [123]//
        \gla 	Hiszen mindig \textbf{siet-ed} {kikér-ni magad-nak}, hogy ál-magyar len-né-l.//
        \glb 	since always hurry-\Tsg.Sbj>\Tsg.\Obj{} protest that fake-Hungarian be-\Cond.\Tsg-\Sbj{}//
        \glft 	\enquote*{Since you always hurry to protest that you're a fake Hungarian.}//
    \endgl
\xe
%\ex\label{ex:kamera-probal}%
%    \begingl
%        \glpreamble \Obj{} --- finite verb --- \Inf{} [312]//
%        \gla 	\nogloss{\dots} még ha csak a kameraszög-et is próbálkoz-od belő-ni.//
%        \glb 	even if only the {camera angle}-\Acc{} too try-\Tsg.\Sbj>\Third.\Obj{} shoot-\Inf{}//
%        \glft 	\enquote*{\dots{} even if you're only trying to get the camera angle right}\trailingcitation{}//
%    \endgl
%\xe
\ex\label{ex:meg-probal}%
    \begingl
        \glpreamble finite verb --- \Inf{} --- \emph{pro} [12\emph{pro}]; \Vm{}-climbing//
        \gla 	\nogloss{\dots} de most teljes üresség van, ha meg
        \textbf{próbálkoz-od} nyit-ni.//
        \glb 	but now complete emptiness \Cop{} if \Vm{} try-\Ssg.\Sbj>\Third.\Obj{} open-\Inf{}//
        \glft 	\enquote*{\dots{} but now it's completely empty if you try to open it}\trailingcitation{}//
    \endgl
\xe

\subsubsection{Third person singular subject, third person object}\label{sub:3-3}

Intransitive predicates are also attested showing agreement with a third person
singular subject and the third person object of their infinitival complement.
The following examples illustrate \emph{készül} and \emph{szándékozik}
\enquote*{to intend}. Analogous constructions with \emph{igyekszik} and
\emph{jár} are also attested in the data set.

%\ex\label{ex:bella-keszul}%
%    \begingl
%        \glpreamble \Obj{} --- finite verb --- \Inf{} [312]//
%        \gla 	Edward Bellá-t készül-i megcsókol-ni//
%        \glb 	Edward Bella-\Acc{} prepare-\Tsg.\Sbj>\Third.\Obj{} kiss-\Inf//
%        \glft 	\enquote*{Edward is preparing to kiss Bella.}\trailingcitation{}//
%    \endgl
%\xe
%\ex\label{ex:lenyul}%
%    \begingl
%        \glpreamble \Obj{} --- \Inf{} --- finite verb [321]//
%        \gla 	\nogloss{\dots} ahogy Sam Witwicky munká-t keres, új barátnő-jé-t, Carly-t lenyúl-ni készül-i annak gyanús főnök-e.//
%        \glb 	as Sam Witwicky employment-\Acc{} seek.\Tsg.\Sbj{} new girlfriend-\Tsg-\Acc{} Carly-\Acc{} {make off with}-\Inf{} prepare-\Tsg.\Sbj>\Third.\Obj{} her suspicious boss-\Tsg{}//
%        \glft 	\enquote*{\dots{} her suspicious boss is preparing to make off with [Sam]'s new girlfriend.}\trailingcitation{}//
%    \endgl
%\xe
\ex\label{ex:birtok}%
    \begingl
        \glpreamble finite verb --- \Inf{} --- \Obj{} [123]//
        \gla 	\nogloss{\dots} birtok-ba \textbf{készül-i} ven-ni az új föld-jé-t.//
        \glb 	possession-\Ill{} prepare-\Tsg.\Sbj>\Third.\Obj{} take-\Inf{} the new land-\Tsg-\Acc//
        \glft 	\enquote*{\dots{} he wants to take his new plot of land into possession.}\trailingcitation{}//
    \endgl
\xe
\ex\label{ex:barat-szandek}%
    \begingl
        \glpreamble \Obj{} --- finite verb --- X --- \Inf{} [312]//
        \gla 	Barát-já-t \textbf{szándékoz-t-a} magá-val vin-ni.//
        \glb 	friend-\Tsg-\Acc{} intend-\Pst-\Tsg.\Sbj>\Third.\Obj{} \Refl.\Tsg{}-\Com{} bring-\Inf{}//
        \glft 	\enquote*{S/he intended to bring his/her friend along.}\trailingcitation{\parencite{Kiss1977}}//
    \endgl
\xe

\subsubsection{First person plural subject, third person object}\label{sub:1pl-3}

First person plural subjects are also found in the relevant constructions,
shown here for \emph{készül} and \emph{siet}, and also attested for
\emph{igyekszik} and \emph{szándékozik}.

\ex\label{ex:meat-market}%
    \begingl
        \glpreamble \Inf --- finite verb --- \Obj{} [213]//
        \gla 	\nogloss{\dots} megválaszt-juk a ruhá-nk-at, megcsinál-juk a
        frizurá-nk-at, az internetes húspiac-on is ugyanúgy elad-ni
        \textbf{készül-jük} magunk-at.//
        \glb 	choose-\Fpl.\Sbj>\Third.\Obj{} the clothes-\Fpl-\Acc{} do-\Fpl.\Sbj>\Third.\Obj{} the hair-\Fpl-\Acc{} the internet.\Adj{} {meat market}-\Supe{} too likewise sell-\Inf{} prepare-\Fpl.\Sbj>\Third.\Obj{} \Refl.\Fpl-\Acc//
        \glft 	\enquote*{\dots{} we choose our clothes, we do our hair, and in the same way we prepare to sell ourselves on the online meat market.}\trailingcitation{}//
    \endgl
\xe
%\ex\label{ex:sajat}%
%    \begingl
%        \glpreamble \Obj{} --- finite verb --- \Inf{} [312]//
%        \gla 	Akár hivatalosak vagy-unk a lagzi-ra, akár a saját-unk-at készül-jük megül-ni \nogloss{\dots}//
%        \glb 	if invited.\Pl{} be-\Fpl{} the wedding-\Subl{} if the own-\Fpl-\Acc{} prepare-\Fpl.\Sbj>\Third.\Obj{} celebrate-\Inf{}//
%        \glft 	\enquote*{Whether we're invited to a wedding, or we are preparing to celebrate our own \dots{}}\trailingcitation{}//
%    \endgl
%\xe
%\ex\label{ex:autoban}%
%    \begingl
%        \glpreamble finite verb --- \Inf{} --- \Obj{} [123]//
%        \gla 	\nogloss{\dots} különösen, ha az autóban készül-jük hagy-ni ő-t.//
%        \glb 	particularly if the car-\Ine{} prepare-\Fpl.\Sbj>\Third.\Obj{} leave-\Inf{} s/he-\Acc//
%        \glft 	\enquote*{\dots{} in particular if we plan to leave her/him [a dog!] in the car.}\trailingcitation{}//
%    \endgl
%\xe
%\ex\label{ex:elet}%
%    \begingl
%        \glpreamble \Obj{} --- finite verb --- \Inf{} [312]//
%        \gla 	\nogloss{\dots} ahol közös élet-ünk elkövetkező minimum 15
%        év-é-t készül-jük le-él-ni.//
%        \glb 	where shared life-\Fpl{} following minimum 15 year-\Tsg-\Pl{}
%        prepare-\Fpl.\Sbj>\Third.\Obj{} \Vm-live-\Inf//
%        \glft 	\enquote*{\dots{} where we prepare to spend at least 15 years of our future life together.}\trailingcitation{}//
%    \endgl
%\xe
\ex
    \begingl
        \glpreamble finite verb --- \Inf{} --- \Obj{} [132]//
        \gla 	Egy-egy ugrás-sal \textbf{siet-t-ük} utolér-ni a civilizáció-ban és a politikai előhaladás-ban a többi európai nemzet-ek-et \nogloss{\dots}//
        \glb 	one-one jump-\Com{} hurry-\Pst-\Fpl{} {catch up}-\Inf{} the civilisation-\Ine{} and the political progress-\Ine{} the other European nation-\Pl-\Acc{}//
        \glft 	\enquote*{We hurried to catch up the other European nations in
        civilisation and political progress with one step or another \dots}//%\trailingcitation{\url{http://www.arcanum.hu/hu/online-kiadvanyok/Mikszath-mikszath-osszes-muve-2A85B/cikkek-es-karcolatok-5186-kotet-40CD5/1880-szegedi-naplo-vezercikkek-es-egyeb-politikai-cikkek-59-kotet-4578F/243-sz-oktober-16-a-fenyuzes-45962}}//
    \endgl
\xe

\subsubsection{Second person plural subject, third person object}\label{sub:2pl-3}

The following examples have second person plural subjects. \eqref{ex:mennyi},
with \emph{készül}, again shows a \emph{pro} object. \eqref{ex:remalom}
and~\eqref{ex:magatokat} are present and past tense examples of
\emph{igyekszik}.

\ex\label{ex:mennyi}%
    \begingl
        \glpreamble finite verb --- \Inf{} --- \emph{pro}//
        \gla 	Mennyi-ért \textbf{készül-itek} ven-ni?//
        \glb 	{how much}-for prepare-\Spl.\Sbj>\Third.\Obj{} buy-\Inf{}//
        \glft 	\enquote*{For how much are you preparing to buy it [a computer]?}\trailingcitation{}//
    \endgl
\xe
\ex\label{ex:remalom}%
    \begingl
        \glpreamble finite verb --- \Obj{} --- \Inf{} [132]//
        \gla 	\nogloss{\dots} azon kívül, hogy igyeksz-itek ez-t a rémálm-ot elfelejt-eni, \nogloss{\dots}//
        \glb 	that apart that strive-\Spl.\Sbj>\Third.\Obj{} this-\Acc{} the nightmare-\Acc{} forget-\Inf//
        \glft 	\enquote*{\dots{} apart from the fact that you strive to forget this nightmare \dots{}}\trailingcitation{}//
    \endgl
\xe
%\ex\label{ex:egymast}%
%    \begingl
%        \glpreamble finite verb --- \Inf{} --- \Obj{} [123]//
%        \gla 	\textbf{Igyeksz-itek} kerül-ni egymást?//
%        \glb 	strive-\Spl.\Sbj>\Third.\Obj{} avoid-\Inf{} {each other}-\Acc{}//
%        \glft 	\enquote*{Are you making an effort to avoid each other?}\trailingcitation{}//
%    \endgl
%\xe
\ex\label{ex:magatokat}%
    \begingl
        \glpreamble finite verb --- \Inf{} --- \Obj{} [213]//
        \gla 	Mi-vel \textbf{igyekez-t-étek} megnyugtat-ni magatokat, amikor elhagyott a szerelmetek több év után?//
        \glb 	what-\Com{} strive-\Pst-\Tsg.\Sbj>\Third.\Obj{} calm-\Inf{} \Refl-\Tpl-\Acc{} when left the love-\Tpl{} several year after//
        \glft 	\enquote*{How did you try to calm yourselves when your lover left you after several years?}\trailingcitation{}//
    \endgl
\xe
%\ex\label{ex:megkerulni}%
%    \begingl
%        \glpreamble finite verb --- \Inf{} --- \Obj{} [123]//
%        \gla 	hogy milyen sokan igyekez-t-étek \nogloss{\dots} megkerül-ni országunk legszebb tav-á-t, a Balatont.//
%        \glb 	that how many try-\Pst-\Spl.\Sbj>\Third.\Obj{} {go around}-\Inf{} country-\Pl{} {most beautiful} lake-\Tsg-\Acc{}, the Balaton-\Acc{}//
%        \glft 	\enquote*{\dots{} how many of you tried to go around our country's most beautiful lake, the Balaton.}\trailingcitation{facebook, Figura Photo, 23 May, 08:53}//
%    \endgl
%\xe

\subsubsection{Third person plural subject, third person object}\label{sub:3pl-3}

Examples with third person plural subjects, agreement with third person
definite objects (\Tpl.\Obj):

%\ex\label{ex:landzsa-keszul}%
%    \begingl
%        \glpreamble \Inf{} --- finite verb --- \Obj{} [213]//
%        \gla 	\nogloss{\dots} a lándzsá-ik-kal épp csikiz-ni készül-ik a hatalmas tappancs-á-t.//
%        \glb 	the spear.\Poss-\Pl-\Tpl-\Com{} just tickle-\Inf{} prepare-\Tpl.\Obj{} the massive paw-\Tsg-\Acc{}//
%        \glft 	\enquote*{They are preparing to tickle his massive paw with their spears.}\trailingcitation{}//
%    \endgl
%\xe
\ex\label{ex:buntetes-keszul}%
    \begingl
        \glpreamble \Obj{} --- finite verb --- \Inf{} [312]//
        \gla 	\nogloss{\dots} a fogadás elvesztés-é-ért járó büntetés-ük-et \textbf{készül-ik} letölt-eni.//
        \glb 	the bet loss-\Tsg-for deserved punishment-\Tpl-\Acc{} prepare-\Tpl.\Obj{} spend-\Inf{}//
        \glft 	\enquote*{\dots{} they were preparing to sit out the punishment they got losing the bet.}\trailingcitation{}//
    \endgl
\xe
\ex\label{ex:szandek-bp}
    \begingl
    \glpreamble \Obj{} --- \Inf{} --- finite verb [321]; \gls{MNSZ}/doc\#901//
        \gla 	\nogloss{\dots{}} hogy valaki-k a Fővárosi Önkormányzat-ot meg-károsít-ani \textbf{szándékoz-zák} vagy \textbf{szándékoz-t-ák} //
        \glb 	that someone-\Pl{} the capital.\Adj{} local.government-\Acc{} \Vm-harm-\Inf{} intend-\Tpl.\Sbj{}>\Third.\Obj{} or intend-\Pst-\Tpl.\Sbj{}>\Third.\Obj{}//
        \glft 	\enquote*{that some people intend or intended to harm the General Assembly of Budapest}//
    \endgl
\xe
%\ex\label{ex:hazas-keszul}%
%    \begingl
%        \glpreamble \Obj{} --- X --- finite verb --- \Inf{} [2X13]; topical \Obj{}//
%        \gla 	Szerelm-ük-et házasság-gal készül-ik megpecsétel-ni//
%        \glb 	love-\Tpl-\Acc{} marriage-\Com{} prepare-\Tpl.\Obj{} stamp-\Inf{}//
%        \glft 	\enquote*{They plan to fulfil their love with marriage.}//
%    \endgl
%\xe
%\ex
%    \begingl
%        \glpreamble finite verb --- \Inf{} --- \Obj{} [132]//
%        \gla 	valami porhintés-sel próbálkoz-zák megvezet-ni az ügyfel-et.//
%        \glb 	something mockery-\Com{} try-\Tpl.\Obj{} mislead-\Inf{} the client-\Acc{}//
%        \glft 	\enquote*{they're trying to mislead the client with some kind of mockery}\trailingcitation{\url{http://kispad.hu/blog/200512/ugyfelszolgalat.html}}//
%    \endgl
%\xe
\ex\label{ex:lovag-igyek}
    \begingl
        \glpreamble finite verb --- \Inf{} --- \Obj{} [123]//
        \gla 	Ezért a német lovag-ok a 14. század-ban \textbf{igyekez-t-ék} elfoglal-ni Litvániá-t is.//
        \glb 	{because of this} the German knight-\Pl{} the 14th century-\Ine{} strive-\Pst-\Tpl.\Obj{} conquer-\Inf{} Lithuania-\Acc{} too//
        \glft 	\enquote*{Because of this, in the 14th century the German knights strove to conquer Lithuania as well.}\trailingcitation{}//
    \endgl
\xe
%\ex\label{ex:fog-igyek}%
%    \begingl
%        \glpreamble \Obj{} --- X --- finite verb --- \Inf{} [3X12]; topical \Obj{}//
%        \gla 	A fog-ak-at már az ókor-ban is igyekez-t-ék pótol-ni.//
%        \glb 	the tooth-\Pl-\Acc{} already the antiquity-\Ine{} too strive-\Pl-\Tpl.\Obj{} replace-\Inf{}//
%        \glft 	\enquote*{They aimed to replace teeth already in antiquity.}\trailingcitation{}//
%    \endgl
%\xe

\section{Distribution of agreement}\label{sec:distribution-of-agreement}

\subsection{Person}

The examples in~\Cref{sec:data} show that object agreement with \emph{third}
person definite objects is found with intransitive verbs like
\emph{próbálkozik}, \emph{készül} and \emph{igyekszik}. This is true for any
combination of person of subject and object where object agreement is overtly
coded.

Each cell in~\Cref{tb:lda-distr} with \ding{51} has at least one attested
instance of agreement with an object of that person with at least one verb.  In
the empty cells in~\Cref{tb:lda-distr}, the verb forms are intransitive anyway,
so there is nothing to look for. Both singular and plural subjects can agree
with \Third{}rd person objects.

\paragraph{Agreement with \Second{}nd and \Third{}rd person}

The difference between agreement with \Second{}nd and \Third{}rd person objects
is gradient, not categorical. In other words, the data shown in~\Cref{sec:data}
indicate that object agreement between intransitive matrix verbs and \Third{}rd
person objects of the embedded infinitival exists, with any subject person ---
cf.~\Cref{tb:lda-distr}.

\renewcommand{\arraystretch}{1.25}
\begin{table}[htpb]
    \centering
    \rowcolors{2}{gray!10}{white}
    \begin{tabular}{llll}
    \toprule
    $\downarrow$ \Sbj{}, \Obj{} $\rightarrow$ & \First{} & \Second{} & \Third{} \\
    \midrule
    \Fsg{}                                    &          & \ding{51}~(\ref{ex:itv-inf}c) & \ding{51}~\eqref{ex:windows-keszul}--\eqref{ex:elhagyni-keszul} \\
    \Fpl{}                                    &          &           & \ding{51}~\eqref{ex:meat-market}--\eqref{ex:elet}\\
    \Ssg{}                                    &          &           & \ding{51}~\eqref{ex:oxford-keszul}--\eqref{ex:meg-probal}\\
    \Spl{}                                    &          &           & \ding{51}~\eqref{ex:mennyi}--\eqref{ex:magatokat} \\
    \Tsg{}                                    &          &           & \ding{51}~\eqref{ex:bella-keszul}--\eqref{ex:barat-szandek} \\
    \Tpl{}                                    &          &           & \ding{51}~\eqref{ex:landzsa-keszul}--\eqref{ex:fog-igyek} \\
    \bottomrule
    \end{tabular}
    \caption{Distribution of LDA with intransitive matrix verbs}\label{tb:lda-distr}
\end{table}

\subsection{Word orders}\label{sub:word-orders}

Examples with overt objects are coded with a permutation of [123], indicating
the order of the finite matrix verb (1), the infinitive (2) and the object (3).

\pex\label{data:word-orders}%
    \a  {}[123]:~\eqref{ex:oxford-keszul},~\eqref{ex:birtok},~\eqref{ex:lovag-igyek}
    \a  {}[132]:~\eqref{ex:remalom}
    \a  {}[213]:~\eqref{ex:meat-market},~\eqref{ex:magatokat}
    \a  {}[231]:
    \a  {}[312]:~\eqref{ex:windows-keszul},~\eqref{ex:barat-szandek},~\eqref{ex:buntetes-keszul}
    \a  {}[321]:~\eqref{ex:szandek-bp}
\xe
The orders [312] and [213] mostly indicate focusing of either the object [312]
or the infinitive [213] in the matrix focus position. Both of these orders lead
to adjacency between the finite verb and the object\footnote{Maybe adjacency
helps construing the object as an argument of the matrix verb;
see~\textcite{Peredy2009} for discussion w.r.t.\ object agreement.} but object
agreement is found without adjacency as well.

[231] order involves fronting both the infinitive, as a (contrastive) topic,
and the object, as a matrix focus; a constructed example with the transitive
verb \emph{akar} is shown in (\nextx):

\ex
    \begingl
        \gla 	Olvas-ni a könyv-et akar-om.//
        \glb 	read-\Inf{} the book-\Acc{} want-\Fsg.\Sbj>\Third.\Obj{}//
        \glft 	\enquote*{As for reading, it is the book I want to read.}//
    \endgl
\xe
I do not see a principled reason for ruling out [231] (as in (\lastx)) with an
intransitive verb like \emph{igyekszik}, \emph{készül}, etc., given the range
of data found with other orders shown in~\eqref{data:word-orders}. However,
checking around 500 examples of \emph{akar} with infinitival complements did
not produce any [231] orders either, suggesting that this order is generally
rare, not just when the matrix verb is intransitive.

\paragraph{Interim summary: word orders}

Out of six possible permutations of the word order of the intransitive finite
matrix verb, an object, and the infinitive, five are readily found. Orders are
influenced by information structure, exhibiting focus and topic movement.
Object agreement between an intransitive matrix verb and the object of the
infinitive is thus not restricted to special word orders or special
configurations of information structure.

\subsection{Past tense}\label{sub:past-tense}

\Textcite{denDikken2004} points out out that with verbs forming
\enquote{come/go verb aspectual constructions} the grammaticality of object
agreement, in particular \Second{}nd person agreement, depends on tense. For
example, \emph{jön} can form a \Fsg>\Second.\Obj{} form in the past but not the
present tense, as shown in (\nextx).

\pex
    \a
    \begingl
        \gla 	Jö-tt-elek meg-látogat-ni \nogloss{(} @ téged\emph{)}.//
        \glb 	come-\Pst-\Fsg>\Second.\Obj{} \Vm-visit-\Inf{} you.\Acc{}//
        \glft 	\enquote*{I came to visit you.}//
    \endgl
    \a
    \begingl
        \gla 	\ljudge*Jö-lek meg-látogat-ni \nogloss{(} @ téged\emph{)}.//
        \glb 	come-\Fsg>\Second.\Obj{} \Vm-visit-\Inf{} you.\Acc{}//
        \glft 	intended: \enquote*{I am coming to visit you.}\trailingcitation{\parencite[451]{denDikken2004}}//
    \endgl
\xe
Other verbs with similar semantics and argument structure, like \emph{jár}
\enquote*{go (regularly)} can form \Fsg>\Second.\Obj{} in both present and
past, although as with all data presented here, there is variation in how
acceptable these forms are:

\pex
    \a
    \begingl
        \gla 	Jár-lak meg-látogat-ni \nogloss{(}téged\emph{)}.//
        \glb 	go-\Fsg>\Second.\Obj{} \Vm-visit-\Inf{} you.\Acc//
        \glft 	\enquote*{I go to visit you regularly.}//
    \endgl
    \a
    \begingl
        \gla 	Jár-ta-lak meg-látogat-ni \nogloss{(}téged\emph{)}.//
        \glb 	go-\Pst-\Fsg>\Second.\Obj{} \Vm-visit-\Inf{} you.\Acc//
        \glft 	\enquote*{I went to visit you regularly.}//
    \endgl
\xe
%I suggest that the reason here is not semantic, but rather morphological:
%past tense stems are always regular, present tense stems need not be.

\paragraph{Preference for past tense: morphology?}

A reason for why past tense forms like \emph{jö-tt-elek}
\enquote*{come-\Pst-\Fsg>\Second.\Obj} are more acceptable than their present
tense counterparts \emph{*jö(l)-lek} \enquote*{come-\Fsg>\Second.\Obj} can lie
in morphology. The present tense forms of \emph{jön}, \emph{megy}, \emph{lenni}
are irregular, while their past tense forms are regular, based on a single stem
ending in \emph{-t}.  It is straightforward to form analogical (agreeing)
patterns based on transitive forms in the past; this is not possible in the
present tense --- cf.~\Cref{tb:jon-prs-pst}.

\begin{table}[h!]%[htpb]
    \centering
    \rowcolors{2}{gray!10}{white}
    \begin{tabular}{lllll}
    \toprule
              & Present        & Past              & Present       & Past \\
    \midrule
    \Fsg{} & \emph{jöv-ök}  & \emph{jö-tt-em}   & \emph{jár-ok}  & \emph{jár-t-am} \\
    \Ssg{} & \emph{jö-sz}   & \emph{jö-tt-él}   & \emph{jár-sz}  & \emph{jár-t-ál} \\
    \Tsg{} & \emph{jön}     & \emph{jö-tt}      & \emph{jár}     & \emph{jár-t} \\
    \Fpl{} & \emph{jöv-ünk} & \emph{jö-tt-ünk}  & \emph{jár-unk} & \emph{jár-t-unk} \\
    \Spl{} & \emph{jöt-tök} & \emph{jö-tt-etek} & \emph{jár-tok} & \emph{jár-t-atok} \\
    \Tpl{} & \emph{jön-nek} & \emph{jö-tt-ek}   & \emph{jár-nak} & \emph{jár-t-ak} \\
    \bottomrule
    \end{tabular}
    \caption{Present and past tense forms of \emph{jön} \enquote*{come}
    (irregular) and \emph{jár} \enquote*{go (regularly)}}\label{tb:jon-prs-pst}
\end{table}

\subsection{Extraction}\label{sub:question-words}

\textcite[34]{KenVagFeny1998} suggest that only predicates like \emph{akar}
\enquote*{want} and \emph{megpróbál} \enquote*{try} can occur with question
words like \emph{mi-t} \enquote*{what-\Acc}. (\nextx) is a counterexample;
note that there is no object agreement here since \emph{mit} never triggers
object agreement.

\ex\label{ex:mit-igyek}%
    \begingl
        \gla 	Mi-t igyekez-t-él elmond-ani nekem?//
        \glb 	what-\Acc{} strive-\Pst-\Ssg.\Sbj{} tell-\Inf{} I.\Dat{}//
        \glft 	cf.\ original \enquote*{What were you trying to tell me?}\trailingcitation{(Paula Hawkins, \emph{Into the Water}; translated by Tomori Gábor)}//
    \endgl
\xe

\subsection{More on \emph{jár}}\label{sub:more-on-jar}

\emph{jár}, while generally intransitive, can be used transitively with
locational objects straightforwardly (also with different \Vm{}s):

\pex
    \a
    \begingl
        \gla 	Jár-om az ut-am.//
        \glb 	go-\Fsg.\Sbj>\Third.\Obj{} the way-\Fsg{}.\Poss{}//
        \glft 	\enquote*{I am going my way.}//
    \endgl
    \a
    \begingl
        \gla 	Jár-ja az ut-já-t.//
        \glb 	go-\Tsg.\Sbj>\Third.\Obj{} the way-\Tsg{}-\Acc{}//
        \glft 	\enquote*{S/he is going his/her way.}//
    \endgl
\xe
In contrast to the predicates in~\Cref{sec:data}, however, it agrees with the
object of the infinitive in even fewer cases. An attested example is shown in
(\nextx).

\ex\label{ex:jarta-nezni}
    \begingl
        \gla 	Két nap-ig a falu nép-e jár-t-a néz-ni a fölakasztott ember-t.//
        \glb 	two day-\Term{} the village people-\Tsg{} go-\Pst-\Tsg.\Sbj>\Third.\Obj{} watch-\Inf{} the hung person-\Acc{}//
        \glft 	\enquote*{The villagers went to watch the hung person for two days.}//
    \endgl
\xe
%Speakers' judgments on similar examples vary: agreeing \emph{készül} seems more
%frequent than agreeing \emph{jár}, but this needs to be checked in detail.

\section{Towards an analysis}\label{sec:analysis}

To the degree that it is accepted, agreement between an intransitive matrix
verb and the infinitive's object is \emph{regular}, i.e.\ a definite second or
third person object can trigger object agreement, but an indefinite object
cannot. This is schematically shown in~\eqref{ex:types}.

\pex\label{ex:types}
    \a 	{}[ V-\Sbj{} \dots{} [\tss{\Inf} V DP-\Def{} ]]\trailingcitation{\faCheck}
    \a 	{}[ V-\Obj{} \dots{} [\tss{\Inf} V DP-\Def{} ]]\trailingcitation{\faCheck}
    \a 	{}[ V-\Sbj{} \dots{} [\tss{\Inf} V DP-\Indef{} ]]\trailingcitation{\faCheck}
    \a 	{}[ V-\Obj{} \dots{} [\tss{\Inf} V DP-\Indef{} ]]\trailingcitation{\faTimes}
\xe
Transitive matrix verbs show types (\ref{ex:types}b,c). Intransitive verbs
can additionally show type (\ref{ex:types}a). But neither class would show
(\ref{ex:types}d), e.g.\ object agreement with an indefinite object.

These patterns can be accounted for with a few assumptions:

\begin{itemize}

    \item 	being intransitive, verbs like \emph{készül}, \emph{próbálkozik},
        \emph{igyekszik} etc.\ do not come with a $\phi$-probe that can agree
        with an \Acc{} object

    \item   but when these verbs appear with infinitival complements, they are
        analogous to transitive verbs like \emph{akar}, \emph{fog} etc.\ which
        agree with a different verb's object

    \item 	this allows speakers to construe the \emph{intransitive}
        verbs as having a $\phi$-probe with infinitival complements

\end{itemize}

\pex[interpartskip=6ex]\label{ex:derivations}
    \a  Agreement with a definite third person object\\
        \dots{} [\tss{\emph{v}P} \tn{v}{\emph{v}[uφ]} [\tss{VP} [\tss{XP} V.\Inf{} \tn{dp}{DP.\Def{}} ]]]
        \td{v}{dp}{\faCheck{} Agree}
    \a  No agreement with an indefinite third person object\\
        \dots{} [\tss{\emph{v}P} \tn{v}{\emph{v}[uφ]} [\tss{VP} [\tss{XP} V.\Inf{} \tn{dp}{DP.\Indef{}} ]]]
        \tx{v}{dp}{\faTimes{} Agree}
    \a  No probe on intransitive \emph{v}\\
        \dots{} [\tss{\emph{v}P} \tn{v}{\emph{v}} [\tss{VP} [\tss{XP} V.\Inf{} \tn{dp}{DP.\Def/\Indef{}} ]]]
\xe
Transitive verbs only allow (\ref{ex:derivations}a,b), but not
(\ref{ex:derivations}c). Intransitive verbs vary: speakers who treat them
analogously to transitives allow (\ref{ex:derivations}a,b), others allow
(\ref{ex:derivations}c), the \enquote{standard} case. But there is no way to
derive a pattern like~(\ref{ex:types}d).

\subsection{Interim summary}

If an intransitive with an infinitival complement has a $\phi$-probe, it
behaves like a regular transitive: if the infinitival complement's object is
definite, it will agree; if the object is indefinite, it will not.

\begin{itemize}

    \item 	agreement with objects of infinitival complements is regular

    \item 	the distribution of probes on intransitive verbs is more
        idiosyncratic

    \item 	intransitives do not have (\Acc{}) objects, but subcategorisation
        for an infinitival complement can add a probe

    \item[\ding{224}]   much of the variation in agreement lies in when
        there can be a probe

    \item[\ding{224}] 	differences between intransitive verbs: regularity of
        paradigms, possibly frequency?

    \item[\ding{224}]   but the variation need not be in the syntax of
        agreement

\end{itemize}

\section{Open questions}\label{sec:open-questions}

\subsection{Second person agreement}\label{sub:2-agr}

\pex
    \a
    \begingl
        \gla 	Hagy-od \nogloss{(} @ János-nak\emph{)} meg-látogat-ni Péter-t.//
        \glb 	let-\Tsg.\Sbj>\Third.\Obj{} János-\Dat{} \Vm-visit-\Inf{} Péter-\Acc{}//
        \glft 	\enquote*{You allow Péter to be visited (by János).}//
    \endgl
    \a
    \begingl
        \gla 	Hagy-lak \nogloss{(*} @ János-nak\emph{)} meg-látogat-ni \nogloss{(} @ téged\emph{)}.//
        \glb 	let-\Fsg>\Second.\Obj{} János-\Dat{} \Vm-visit-\Inf{} you.\Acc//
        \glft 	\enquote*{I let you be visited (by János).}\trailingcitation{\parencite[453]{denDikken2004}}//
    \endgl
\xe
\Textcite{denDikken2004} suggests that the obligatory absence of
\emph{Jánosnak} in (\lastx b) indicates that agreement with a third person
and agreement with a second person are different, since \emph{-lak/-lek} is
blocked by the intervening \Dat{}.

\subsubsection{Individual speakers?}

Another question is why and to what degree speakers \emph{only} accept second
person agreement but not third person agreement (or other patterns). This
question has to be looked at by studying individual speakers in depth. The data
above are more general and describe inter-speaker variation.

\section{Conclusions}\label{sec:conclusions}

Intransitive verbs with infinitival complements can agree with the object of
the infinitive.

\begin{itemize}

    \item   Agreement is found with third person definite objects and
        with second person objects

    \begin{itemize}

        \item 	with any subject person

        \item   with different predicates

        \item   with different information structures and word orders

        \item 	in the present and the past tense alike

    \end{itemize}

    \item[\ding{224}]   Object agreement need not be categorically different
        with third and second person objects

    \item   Competing grammatical pressures can motivate agreement on
        intransitives

    \begin{itemize}

        \item 	intransitive verbs do not select for objects: \emph{contra} agreement

        \item 	analogy to transitive verbs in the same contexts: \emph{pro} agreement

    \end{itemize}

\end{itemize}

\section*{Acknowledgments}

Thanks to Krisztina Szécsényi for discussion, and Júlia Keresztes and Ádám
Szalontai (as well as a few other native speakers) for judgements.

\newrefcontext[sorting=nyt]
\printbibliography

\appendix

\section{Sources}\label{app:sources}

See \url{http://github.com/andrasbarany/icsh13/} for the full data set which
also includes the search terms used in the \gls{MNSZ} as well as their document
identifiers, and URLs for the data from other sources on the internet.

\end{document}
